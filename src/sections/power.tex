\section{Power}

In this section I'm going to outline the design of and choices involved in creating a resilient electrical system with no single point of failure that is capable of enabling someone to work indefinitely.

Power will be standardised at 12v as well as the USB-C Power Delivery standard for use with laptops.

It's very fortunate that it's now practical to entirely forgo 220v AC when powering an office; there is one fewer point of failure and efficiency is increased hugely. This efficiency saving can be anywhere from 10 to 40 \% depending on operating conditions, this is a big win.

\subsection{out}
I have identified the following power requirements, and have assumed 24 hour always on to provide an extra buffer to take into account such extreme eventualities, ensuring that it's always possible to work no matter what.

\subsubsection{power tools}
This might be an area requiring an alternator

\begin{itemize}
    \item fridge ()
    \item freezer
    \item lights
    \item workstation
    \item monitors
    \item navigation electronics
    \item watermaker (reverse osmosis / desalination machine)
    \item water heater
    \item compute/networking
    \begin{itemize}
        \item satellite
        \item 5g
        \item 4g
        \item WiFi
        \item server
    \end{itemize}
\end{itemize}

\subsubsection{Marine requirements}

Without going into too much detail, the problem of electrifying a boat is a solved issue, in that sense I refer the reader to such experts \cite{boat-power-1} \cite{boat-power-2} \cite{boat-power-3} who estimate the maximum power draw whilst at sea (which is when the most power is used) over a 24 hour period to be about 3000 Watt hours. In contrast, when at anchor (The situation I anticipate being in on a daily basis during the work week) this reduces by around 2/3rds to 1100 Watt hours. These figures include very power hungry items such as an electric autopilot which I intend not to use (in this case I'd prefer a wind-vane).

This figure takes into account all electronics except those required for work, which I do below.

\subsection{Office power}

My work-from-home set up currently consists of a macbook driving two 4k 27-inch monitors, with an external mouse and keyboard with a standing desk. I will be replicating this set up in the boat. Not compromising on a workstation is absolutely non-negotiable for me and I discuss my workstation twice, here where I talk about the power requirements and below in the logistics section.

I will assume that the equipment involved will be used for 12 hours a day to simulate a particularly busy day, and perhaps with an outage that must be attended to outside of normal working hours.

\subsubsection{Monitors}
Each monitor draws 24 Watts nominally, with a peak power draw of 80W. Having installed a kill-a-watt I am unsure how this 80 Watt figure was reached since I couldn't replicate such a high draw under any circumstances. The highest I could achieve was 42 Watts and as such will use that figure. Two monitors drawing 42 Watts each for an extended period of 12 hours would consume a total of 1008 Watt hours. At a nominal draw this would consume a total of 576 Watt hours.

\subsubsection{Laptop}
I have to make some assumptions about computer hardware since I will not be in control of what I get, so I will assume an upper bound and behave accordingly. A 2020 Intel based 16-inch Macbook pro has a peak power draw of ~40 Watts, and is the most power hungry Macbook that's currently available. Similarly to the monitors it does not reach this 40 Watt load unless it is working at 100\%. As an upper limit, 40 Watts for 12 hours of use gives a total daily power usage of 480 Watt hours, this is far above what I would expect in actuality, since such a machine typically idles at around 15 Watts, implying a daily use of 180 Watt hours.

\subsubsection{Networking}



\subsubsection{Total}
In total, as an upper bound the office equipment would require




\subsection{Storage}
The energy storage solution is going to be well oversized

\subsection{In}
I've identified the following sources of power
\begin{itemize}
    \item solar
    \item wind
    \item saildrive
    \item emergency diesel generator
\end{itemize}

\subsection{Peak power use}
In all honesty, it's a stroke of luck that unlike a conventional power grid utilising solar power, which has peak supply in the middle of the day and peak use in the evening, this grid will have overlapping peak supply (solar power during the day) and draw (office equipment).

This has the implication that the external monitors (identified as being the most power hungry item on the boat after the desalination machine) don't require that their power budget is calculated in the same way as most other items since we can safely assume that they will be running directly off of the solar during the vast majority of their use. This is a big win, essentially allowing me to half the size of the battery bank.

Similarly the desalination machine can be used whenever needed, so with minimal planning one can utilise what would otherwise be over-charge (i.e. electricity that would be generated but cannot be stored since the batteries are full) for producing water.

\subsection{Redundancy}
There are several levels at which the power grid must be made redundant to ensure redundancy and durability of thegrid itself, these are:
\begin{itemize}
    \item generation
    \item charging
    \item storage
    \item utilisation
\end{itemize}

With these taken care of, I will achieve a resilient power-grid capable of supplying enough power to work indefinitely and independently of any external sources.

\subsubsection{Generation}
Power generation will be handled primarily by multiple solar panels, with wind and ocean currents being used to even out generation - wind and solar generation in particular are well known to be negatively correlated \cite{wind-solar-correlation}, meaning that when one is producing less power, the other is likely producing more.

Due to the majority of power being generated by solar we will handle redundancy within solar generation as well as across generation types. This is handled quite simply by stringing panels up in parallel so that, given N panels, if any single panel fails we lose 1/Nth of our generation capacity. For example, with 5 panels strung in parallel we would lose 20\% of our solar generation capacity if one were to fail. Independent of the actual amount of generation required, using sufficiently many solar panels ensures that a 1/Nth drop in production does not mean eventually out of power before being able to replace the panel, I anticipate N = 4 is sufficient.

\subsubsection{Charging}
Charge controllers are getting to a point today where one no longer needs separate solar, wind and motor charge controllers, drastically simplifying their implementation. Instead we can use a single, one size fits all solution. 

To achieve high availability at this level we will use two charge controllers in parallel, where each is sufficiently sized to handle the maximum possible charge in the case that the other were to fail.

\subsubsection{Storage}
House and Motor power will be stored in a single, large battery bank. We can achieve redundancy of power but not high availability, damage to or loss of a battery \textit{will} mean a temporary outage for the network but any work in progress won't be affected due to laptops having an inbuilt UPS.

In the event of damage to or loss of a cell, the topology of the battery bank means that restarting the system will be as simple as isolating the battery from the bank and resetting the circuit breaker. In this way we will lose 1/Nth maximum current draw, in an identical manner to the discussion on solar panels.

\subsubsection{Utilisation}

Since the vast majority of the boat will be standardised to 12v DC no component is needed to interface the battery bank with any of the electronics that utilise it, with the exception of USB C power delivery devices as discussed below. Having no component here means it can't fail, nice!

In the case of devices charged using the USB power delivery standard, we will simply use multiple controllers. They nominally have 1 USB-C power delivery port and so multiple will likely be required around the boat in any case. Due to the nature of anything being charged in this manner having a battery, I have no concerns about ensuring 100\% up-time and would keep a cold spare. 

\subsection{Emergencies}
In an effort to minimise downside risk under any circumstance I will have a small diesel generator on standby with enough fuel for 100\% utilisation for at least a week. This is incredibly conservative, and doubles as redundant power for the electric motor if all other safeguards fail.

\subsection{System Degradation}
Solar panels and batteries lose their capabilities as they age, this degradation must be taken into account so that the power solution now is still suitable later on.

Lithium iron phosphate (LiFePO4) batteries for example are generally rated to have 80\% of their original capacity after 5000 discharge cycles, that's a little over 13 years.

Solar panels have a much longer lifespan with 90\% generation typically guaranteed after 10 years, and 80\% generation typically guaranteed after 20 years. This of course will vary from manufacturer to manufacturer. 

Given these time frames I won't go into more detail about the long term viability of this solution. Having to worry about this in real terms would be a very welcome problem indeed and can be easily solved by getting an additional solar panel wired in parallel with the rest ~10 years down the line.


\section{Wiring diagram}

Much like a distributed system, the power system must have 